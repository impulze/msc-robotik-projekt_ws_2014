\documentclass[a4paper]{article}

\sloppy

\usepackage[T1]{fontenc}
\usepackage[utf8x]{inputenc}
\usepackage[pdftex]{graphics}
\usepackage[ngerman]{babel}
\usepackage{graphicx}
\usepackage{tabularx}
\usepackage{amsmath,amssymb,amstext}
\usepackage{bibtopic}
\usepackage{float}
\usepackage{hyperref}
\usepackage{multirow}
\usepackage{listings}

\author{Daniel Mierswa}

\renewcommand{\lstlistingname}{\normalsize Quellcode}
\renewcommand{\lstlistlistingname}{Verzeichnis der Quellcodes}

\pagestyle{plain}
\bibliographystyle{alpha}

\newcommand{\cmacro}[1]{\mbox{\textsc{#1}}}

\usepackage{pdfpages}

\title{
	\textbf{Dokumentation für das Projekt in der Lehrveranstaltung Robotik}
}

\author{
	\textit{Mierswa, Daniel} \\
	Matrikelnummer: XXXXXX \\
	daniel.mierswa@student.hs-rm.de \\
}

\begin{document}

\maketitle
\newpage
\tableofcontents
\newpage

\section{Einleitung}

Die Robotik ist längst ein großes Anwendungsfeld der Informatik.
Roboter, als Entitäten der Robotik, werden z.B. stationär in der Industrie
oder autonom in Gefahrengebieten genutzt.
Dadurch wird ersichtlich, dass die Einsatzbereiche sehr heterogen sind und
entsprechend andere Anforderungen an einen Roboter existieren.
In Zukunft werden Roboter, auch bedingt durch die wachsende Anzahl an eingebetteten
Geräten, die einem im Alltag begegnen, vermehrt in anderen Gebieten erwartet, z.B. im Pflegebereich
(Ambient Assited Living) oder Servicebereich, wodurch sie auch Serviceroboter genannt werden.
Bei stationären Robotern ist die Umsetzung der Kinematik ein wesentlicher Aspekt und wird
in der Regel durch entsprechende Mathematik unterstützt.
Für alle Robotertypen ist die Navigation, also insbesondere die Wegfindung bzw. Bahnplanung und
die damit verbundene Kollisionsprüfung eine zu lösende Herausforderung und ein Anwendungsgebiet der Informatik.
Für die Bahnplanung existieren verschiedene Konzepte, inklusive derer Modelle und Algorithmen.
Aufgrund der vielfältigen Möglichkeiten und Lösungsansätze zur Planung und Umsetzung eines
mobilen Roboters, entstand im Rahmen der Lehrveranstaltung Robotik im Master-Studiengang Informatik
der Hochschule RheinMain, ein Projekt, welches einen konkreten Anwendungsfall umsetzen soll.

\section{Aufgabe}

Die Aufgabe, die dieses Projekt versucht zu lösen, umfasst die Bahnplanung in zwei nahe zu
gleichgroßen 2D-Räumen.
Die Räume werden mit festen Blockaden bzw. Gegenständen bestückt und einer Tür bestückt.
In diesen Räumen sollen statistisch Wegpunkte verteilt werden, welche dann als Knoten für einen gewichteten Graphen dienen.
Anhand dieses Graphen, soll dann mit dem Dijkstra-Algorithmus der kürzeste Pfad von einem
Start- zu einem Endpunkt berechnet werden.
Falls genügend Bearbeitungszeit zur Verfügung, kann als zusätzliche Alternative der A*-Algorithmus
verwendet werden.
Nachdem ein Pfad gefunden wurde, soll geprüft werden, ob dieser mit den Wänden des Raumes oder den darin befindlichen
Objekten kollidiert.
Falls keine Kollision auftritt, wird mittels der Anwendung von Catmull-Rom Splines eine Kurve
erstellt, die die entsprechenden Wegpunkte berührt.
Erneut wird nach dem Erstellen der Kurve geprüft, ob Kollisionen auftreten.
Zur Veranschaulichung der Lösung, soll ein Programm entwickelt werden, welches mindestens die folgenden Funktionalitäten anbietet:

\begin{itemize}
\item Darstellung der 2D-Räume und der darin befindlichen Objekte
\item Roboter-Entität als kreisförmiges Objekt
\item Eingabe des Start- und Zielpunkts
\item Verändern der Anzahl der Wegpunkte
\item Anzeige der Programmlaufzeit für die Bahnplanung
\item Veranschaulichung einer möglichen auftretenden Kollision
\end{itemize}

Aufgrund der Anforderungen müssen im Programm folgende Aspekte umgesetzt werden:

\begin{itemize}
\item Verarbeitung einer Bilddatei eines Raumes
\item Erstellen einer Benutzeroberfläche
\item Zeichnen einer 2D-Szene inklusive entsprechenden Veränderungen, die durch den Benutzer gesteuert werden
\item Messen von Programmlaufzeiten für die Durchführung verschiedener Aktionen
\end{itemize}

\section{Grundlagen}
\label{Grundlagen}

Um die im weiteren Verlauf beschriebenen Lösungsansätze besser zu verstehen, werden in diesem Kapitel
die Begriffe, die in der Aufgabenstellung relevant sind, näher erläutert.

\subsection{Bahnplanung}

Um eine Bahnplanung programmatisch zu realisieren, werden Modelle erstellt, in denen dann Algorithmen
einen Pfad finden können.
Die Modelle werden entweder statisch hinterlegt oder dynamisch zur Laufzeit berechnet.
Je nach Einsatzgebiet des Roboters empfiehlt sich die eine oder andere Methode.
Auch vom Einsatzgebiet und der Aufgabenstellung des Roboters ist abhängig, ob ein 2D- oder 3D-Modell
erstellt werden muss.

Eine mögliche 2D-Modellierung ist das Umgebungsmodell.
In einem Umgebungsmodell wird die Umgebung in ein Raster eingeteilt.
Die Zellen in dem Raster können dann entweder frei, besetzt oder, bei einem groben Raster, teilbesetzt sein.
Als Hilfestellung dienen oft CAD-Pläne des Raumes (oder Industriehalle, o.ä.).
Der Roboter wird dann entsprechend seiner Form in dieses Modell projiziert und die Aufgabe ist es dann,
den Roboter in diesem Umgebungsmodell so navigieren zu lassen, dass er keine belegten Zellen berührt,
um so eine Kollisionsprüfung zu vermeiden.
Im Falle einer Berührung mit einer teilbesetzten Zelle, muss eine Kollisionsprüfung vorgenommen werden.

Eine Alternative zum Umgebungsmodell ist das Konfigurationsmodell.
Im Konfigurationsmodell wird der Roboter als Punkt (in der Bildverarbeitung z.B. ein Pixel) modelliert.
Die Objekte im Raum werden entsprechend des Durchmessers des Roboters mit einem Rand versehen.
Die Betrachtung dieses Modells erlaubt eine einfachere Bahnplanung.
Der Unterschied der beiden Modelle kann analog in der Bildverarbeitung bei der Erosion und Dilation
wiedergefunden werden.

\subsection{Dijkstra-Algorithmus}

Der Dijkstra-Algorithmus wurde 1956 vom Wissenschaftler Edsger Dijkstra entwickelt.
Er dient zur Findung des kürzesten Pfades in einem gewichteten Graphen mit positiven
Kantengewichtungen.
Der Algorithmus ist \emph{greedy}, was bedeutet, dass er das beste Ergebnis
in jedem Schritt betrachtet, um so eine Gesamtlösung zu formen.
Die Knoten werden mit einer Distanz versehen, welche sich aus verschiedenen Schritten errechnet.
Initial werden alle Knoten mit einer Maximaldistanz (\emph{Unendlich}) versehen.
Der Startknoten bekommt die Distanz \emph{0}.
Weiterhin wird initial eine Menge erstellt, in der alle Knoten (ohne den Startknoten) enthalten sind.
Nun werden alle noch nicht betrachteten Knoten aus dieser Menge iterativ betrachtet.
Dabei wird immer zunächst der Knoten, mit der minimalen Distanz gewählt.
Bei der Betrachtung wird die Distanz neu berechnet, indem die Gewichtung der Kante mit einbezogen wird.
Dieser Knoten wird dann mit der kleinsten Distanz markiert und im Algorithmus nicht weiter betrachtet.
Ist man am Endknoten angelangt, so kann der Algorithmus hier terminieren, wenn nur der kürzeste
Weg zwischen einem Start- und Endknoten gesucht wird.

\subsection{A*-Algorithmus}

Der A*-Algorithmus ist im wesentlichen eine Erweiterung und Verbesserung des Dijkstra-Algorithmus.
Er wurde im Stanford Forschungsinstitut 1968 erstmalig beschrieben.
Um die Laufzeit zu verbessern, benutzt der A*-Algorithmus eine heuristische Funktion.
Genau wie beim Dijkstra-Algorithmus, wird aus der Menge der Knoten, welche noch nicht betrachtet wurden,
ein Knoten zur Betrachtung ausgewählt.
Dieser Knoten wird so ausgewählt, dass die heuristische Funktion (oft die euklidische Distanz)
minimal ist.
Anders formuliert ist der Dijkstra-Algorithmus eine Spezialisierung des A*-Algorithmus, wobei
die heuristische Funktion stets einen festen Wert (\emph{0}) liefert.
Damit der Algorithmus optimal ist, darf die heuristische Funktion keine größere Distanz zwischen
zwei Knoten liefern, als die kleinste tatsächliche Distanz zwischen ihnen.

\subsection{Catmull-Rom Splines}

%http://www.cemyuksel.com/research/catmullrom_param/catmullrom.pdf
%http://www.cs.utexas.edu/~fussell/courses/cs384g/lectures/lecture16-Interpolating_curves.pdf

Catmull-Rom Splines ist eine Klasse von interpolierenden Polygonzügen, die von Edwin Catmull und
Raphael Rom 1974 entwickelt und beschrieben wurde.
Sie werden in der Modellierung bis zur Animation verwendet.
Interpolierende Polygonzüge berühren dessen Stützpunkte, was zur Folge hat, dass bei der Benutzung
dieser Splines eine Kontrolle über einige Punkte des Splines erlaubt.
Ebenfalls hat der Catmull-Rom Spline lokale Eigenschaften, so dass bei der Änderung an bestimmten
Kontrollpunkten nur ein bestimmtes Teilsegment der Kurve beeinträchtigt wird.
Catmull-Rom Splines sind weiterhin glatt, was für die Animation eine relevante Eigenschaft ist.
In der Klasse der Splines unterscheidet man zwischen verschiedenen Parametrisierungen, so dass
die Stützpunkte in verschiedener Weise durchlaufen werden.

Der Spline ist kubisch, was bedeutet, dass 4 Berechnungspunkte für eine Kurve (bzw. ein Kurvensegment)
benötigt werden, welche durch 2 Stützpunkte verläuft.
Die zusätzlich benötigten 2 Stützpunkte enthalten den vorherigen Punkt und den nächsten Punkt der Kurve.
Da es im letzten Kurvensegment keinen nächsten Punkt bzw. im ersten Kurvensegment keinen vorherigen Punkt
gibt, müssen diese beiden Segmente separat betrachtet werden.
Die Punkte der Verbindungslinie können mit linearer Algebra berechnet werden.

\section{Konzept}
\label{Konzept}

Im folgenden Kapitel wird das Konzept des Programms näher erläutert.
Zunächst wird die Architektur und die Komponenten des Programms erklärt.
Für jede Komponente folgt eine zusätzliche Beschreibung, in der feiner erklärt wird,
wie die entwickelten Teillösungen die Lösung der Aufgabenstellung unterstützen.

Das Programm besitzt im wesentlichen 3 Teilsysteme: Graphische Bedienoberfläche, Algorithmen und Datenverarbeitung, sowie
ein Teilsystem, welches die Szene in der sich der virtuelle Roboter bewegt, visualisiert.
Sofern möglich, wurde versucht, die einzelnen Systeme modular in Programmcode zu entwickeln.

%\includegraphics{arch.png}

\subsection{Graphische Bedienoberfläche}

Das GUI wird mit Qt entwickelt.
Die Oberfläche besteht aus einer 2-teiligen Menüführung (wie üblich unter der Titelzeile) und
einem 2-spaltigen Layout als zentrale Komponente.
Die Menüführung dient zum einen dazu, das Programm zu beenden und zum anderen dazu, den aktuellen
Stand des Programms in eine Datei zu speichern, so dass dieser zu einem späteren Zeitpunkt
nachgeladen werden kann.
Ein weiterer Menüpunkt wird bereitgestellt, um ein Bild eines Raumes zu laden, worin der Roboter
seine Bahnplanung durchführen soll.

Wurde ein Bild eines Raumes erfolgreich geladen, so wird in der rechten Spalte der Oberfläche eine Szene
angezeigt, in der sich verschiedene Elemente befinden.
In der linken Spalte befinden sich verschiedene Auswahlkästchen, mit denen die Szene beeinflusst werden kann.
Um die Elemente in der Szene besser sichtbar zu mache, werden diese unterschiedlich eingefärbt.

\subsection{Visualisierung}

Für die Visualisierung wurde das OpenGL-Framework benutzt, welches es ermöglicht sehr detaillierte Szenen
zu konstruieren.
Das Framework war für das Projekt nicht zwingend erforderlich, wurde aber dennoch gewählt, da es bereits
in vorherigen Lehrveranstaltungen, die der Projektdurchführende besucht hat, verwendet wurde und somit eine gewisse
Erfahrung vorhanden war.
Als Alternative hätten ebenfalls Komponenten des GUI (z.B. QGraphicsScene o.ä.) verwendet werden können.

Das Framework bietet mit der Verarbeitung von Texturen und dem Zeichnen von Punkten und Linien, 2 wesentliche Komponenten,
die zur Bearbeitung des Projekts benötigt wurden.
Entsprechend der Einstellungen aus der graphischen Bedienoberfläche, werden in der Szene Aktionen ausgeführt.
Dadurch muss das Modul der Visualisierung eine Schnittstelle anbieten, die von der Bedienoberfläche benutzt werden kann.
Die Schnittstelle umfasst das Setzen von Optionen (entsprechend jener aus der Bedienoberfläche), das Abfangen
von Mausklicks, das Laden von einer Raum-Bilddatei, sowie die grundlegende Schnittstelle zum Initialisieren der Graphikszene
und dem Zeichnen der aktuellen Szene.

\subsection{Algorithmen zur Bahnplanung}

Die Algorithmen, welche zur Berechnung der notwendigen Punkte gebraucht werden, befinden sich in extra
Modulen.
Sie werden bei der Visualisierung einmalig verwendet und die Ergebnisse im Programmspeicher hinterlegt.
Dies hat zur Folge, dass der Speicherverbrauch etwas wächst, jedoch die Performance stark positiv
beeinflusst, da die  Berechnungen sonst bei jedem Neuzeichnen der Szene (die Visualisierungskomponente zeichnet
die Szene stets neu) durchgeführt werden müssten.
Für die Module existieren 2 große Teilmengen von Algorithmen: Algorithmen zur Planung eines Pfades durch
verschiedene Punkte und Algorithmen zur Verarbeitung einer Bilddatei.

\section{Umsetzung}

Im folgenden Kapitel wird die konkrete Umsetzung basierend auf dem vorher entwickelten Konzept \ref{Konzept}
detailliert beschrieben.
Dabei wird zentral betrachtet, wie eine Problemstellung versucht wurde zu lösen, insbesondere
welche Lösungsansätze es gibt und warum sich für eine bestimmte Lösung entschieden wurde.
Weiterhin werden Algorithmen, welche für die Ausführung benötigt wurden und im Grundlagen-Kapitel \ref{Grundlagen} nicht
erklärt wurden, sofern als sinnvoll erachtet, beschrieben.

\subsection{Bildverarbeitung}

Für die Verarbeitung des Raumes wird nicht jeder Pixel des Bildes benötigt.
Vielmehr wird das Bild so analysiert, dass die Rahmen der darin befindlichen Objekte, sowie der Rahmen
des Raumes und der darin befindlichen Türen separat gespeichert werden.
Damit die Algorithmen zur Analyse und Verarbeitung korrekt funktionieren, müssen die Rahmen und Wände
jeweils geschlossen sein und die Tür rechteckig im Bild vorhanden sein.
Eine Beispieldatei liegt dem Programmcode bei.
Es folgt eine Erklärung der einzelnen Schritte der Bildverarbeitung.

\subsubsection{Bildeigenschaften}
\label{Bildeigenschaften}

Damit das Programm ein Bild überhaupt laden kann, muss es gewisse Eigenschaften aufzeigen.
Es werden nur Bilder vom Typ \emph{Portable Network Graphics} (kurz: \emph{PNG}) akzeptiert.
Diese dürfen nur die beiden Farbmodelle RGB oder RGB mit Alpha-Komponente enthalten.
Eine solche Bilddatei wird mittels der Bibliothek \emph{DevIL} in den Programmspeicher geladen.

Um eine Verarbeitung des Bildes im Kontext eines Raumplans zu ermöglichen, müssen die Wände und
Rahmen von Hindernissen bzw. Objekten im Raum schwarz sein (RGB-Komponenten jeweils auf Minimum).
Eine weiße Farbe (RGB-Komponenten auf Maximum) bedeutet, dass sich dieser Pixel des Bildes
außerhalb des Raumes befindet.
Die Farbe Grau (RGB-Komponenten auf 200 bei einer Skala von 0-255) bedeutet, dass dieser Pixel
Teil einer Tür des Raumes ist.
Dies ist im weiteren Verlauf notwendig, damit ohne weitere Algorithmen erkannt wird, wo sich eine
Tür im Raum befindet, damit genau in der Mitte einer Tür ein Wegpunkt gesetzt werden kann.
Jegliche andere Farbe bedeutet, dass sich dieser Pixel im Raum befindet.
Somit ist es z.B. möglich den Boden des Raums einzufärben.

\subsubsection{Rahmenpolygone des Raumes}
Die Schnittstelle zum Modul der Bildverarbeitung bietet nun eine Funktion, um das Rahmenpolygon
des Raumes, sowie der darin befindlichen Objekte und der Türen zu erhalten.
Im 1. Schritt wird eine Abbildung erzeugt, welche für jede Koordinate im Bild speichert,
welche Position sie im Raum entspricht (Im Raum, Außerhalb des Raumes, Tür oder Wand).
Dann wird für jeden Pixel, der zu einer Tür oder sich im Inneren des Raumes befindet geprüft,
ob sich in der nähe eine Wand befindet.
Dazu wird geprüft, ob sich innerhalb des Radius des Roboters eine Wand befindet, falls ja, wird
dieser Pixel aus der Menge der zu betrachtenden Knoten entfernt.
Der letzte Schritt umfasst das Expandieren der noch zu betrachtenden Pixel.

Jeder Pixel, der noch zu einer Tür oder sich im Raum befindet und nicht mit einer Wand kollidiert,
falls sich an dieser Raumposition der Roboter befindet, wird nun expandiert.
Das Expandieren bedeutet, dass von der betrachtenden Position in eine Richtung gewandert wird und
geprüft wird, ob sich dort ein Nachbarpixel befindet der ebenfalls den gleichen Typ repräsentiert
(Tür oder Innenseite des Raumes).
Falls nicht wird die Richtung geändert, in der die Pixel betrachtet werden.
Genau dieser Pixel, an dem die Richtung geändert wird, ist somit ein Eckpunkt eines Polygons.
Befindet man sich Ende der betrachteten Pixel, so ist das Polygon geschlossen.
Das Resultat dieser Schritte ist eine Liste von Polygon, die man als Begrenzungen (\emph{Constraints})
des Raumes auffassen kann.

Diese Liste ist hilfreich um zu prüfen, ob Wegpunkte innerhalb des Raumes sind oder um sie dort
statistisch zu verteilen.
Weiterhin werden die Kanten der Raumpolygone benötigt, um zu prüfen, ob der berechnete Pfad mit einem
Objekt im Raum oder einer Wand kollidiert.

\subsection{Bedienoberfläche}
\label{Bedienoberfläche}

Das GUI wurde mit dem Framework Qt in der Vorabversion 5.4 entwickelt.
Zunächst wurde mit der stabilen Version 4.8 programmiert, was sich später jedoch als nicht weiter
durchführbar herausstellte, da für das Visualisieren des Roboters eine Komponente (\emph{QOpenGLWidget})
benötigt wurde, welche erst in der Vorarbversion verfügbar war.

Wurde ein Bild über die Menüoption \emph{Project -> Load Room} geladen, welches die geforderten
Bildeigenschaften \ref{Bildeigenschaften} erfüllt, so wird die linke Spalte etwas schmaler und auf
der rechten Seite erscheint das Bild des Raumes.
Für das geladene Bild wird zufällig ein Start- und Zielpunkt für den Roboter gewählt, welcher
sich im Raum befindet.
Die generierten Start- und Zielpunkte werden entsprechend grün und rot in der Szene als runder
Kreis gemalt.

Auf der linken Seite befinden sich Bedienelemente für die Szene:
\begin{itemize}
\item Optionen zur Modifizierung von Wegpunkten
\item Auswahlkästchen zur Anzeige von Überprüfungshilfen
\item Ein Knopf zum Animieren des Roboters
\item Auswahlkästchen zur Wahl des Algorithmus zum Berechnen des Pfades
\item Ein Textkästchen für Statusnachrichten
\item Ein Textkästchen für Hilfestellungen
\end{itemize}

Derzeit steht an Algorithmen zur Pfadberechnung der Dijkstra- und A*-Algorithmus
(siehe Grundlagenkapitel \ref{Grundlagen}) zur Auswahl.
Das Textkästchen für Hilfestellungen gibt eine Kurzhilfe, wenn die Maus über eines der Elemente
der linken Spalte schwebt.
Das Textkästchen für Statusnachrichten wird zur Programmlaufzeit Text enthalten, welcher
insbesondere Fehler beschreibt.
Die Statusnachrichten bleiben bis zur nächsten Statusnachricht enthalten, Hilfestellungen
verschwinden beim Verlassen der entsprechenden Schwebefläche mit der Maus.

\subsubsection{Modifizieren von Wegpunkten}

Es ist möglich Wegpunkte interaktiv im Raum zu verteilen.
Dazu muss entsprechend in der linken Spalte eine entsprechende Wegpunkt-Option ausgewählt werden
und im Raum mit der linken Maustaste geklickt werden.
Das Programm fängt den Mausklick ab und überprüft, welches Auswahlkästchen gerade gesetzt ist
und führt die entsprechende Operation durch.
Sollten sich Wegpunkte überlappen bzw. wird versucht Wegpunkte zu löschen, die nicht vorhanden sind,
wird eine Fehlermeldung als Statusnachricht ausgegeben.
Startpunkt und Endpunkt können ebenfalls über diese Methode gesetzt werden.
Gleich zu Beginn, also nach dem Laden des Bildes, werden Wegpunkte in der Mitte von Türen erzeugt.
Die aktuelle Anzahl von Wegpunkten befindet sich rechts neben der Kennzeichnung \emph{Amount of nodes}.

Dieses Kästchen hat eine weitere Eigenschaft.
Wird dort eine Zahl eingegeben, so werden statistisch Wegpunkte im Raum verteilt.
Dieses Vorgehen löscht alle vorher gesetzten Wegpunkte.
Wegpunkte in der Mitte der Türen bleiben erhalten, somit kann die Zahl die in dem Feld erscheint höher
sein, als jene die eingegeben wurde.
Sollten die Wegpunkte der Türen nicht gewünscht sein, müssen sie interaktiv gelöscht werden.

\subsubsection{Einstellungen und Überprüfungen der Szene}

Unter den Modifizierungen befinden sich Optionen, die die Ausgabe der Szene beeinflussen.
Die erste Option \emph{Show triangulation} zeigt eine Triangulierung der Wegpunkte, inklusive
des Start- und Endpunkts.
Diese wird in einer hellblauen Farbe in der Szene angezeigt.
Die Option \emph{Show room triangulation} sorgt dafür, dass die komplette Triangulierung der Rahmenpolygone, die
zuvor ermittelt wurden, in einer etwas helleren blauen Farbe in der Szene erscheint.
Je nach Komplexität des Raumes kann dies das Zeichnen von vielen Linien nach sich ziehen, wodurch diese
Option ausschließlich zu Testzwecken genutzt werden sollte.
Möchte man Wegpunkte oder den generierten Pfad angezeigt bekommen, so setzt man einen Haken in die entsprechende
Optionen \emph{Show waypoints} oder \emph{Show path}.
Die Wegpunkte werden als gelbe Punkte in der gleichen Größe wie Start- und Zielpunkt gezeichnet.
Eine letzte Option \emph{Show neighbours} dient dazu, sich die Nachbarn von Wegpunkten anzeigen zu lassen und
zusätzlich darstellen zu lassen, ob eine direkte Verbindung zum Nachbarn mit einem Raumobjekt oder einer Wand
kollidiert.
Um diese Funktion nutzen zu können, müssen die Auswahlkästchen in den Modifizierungen (oben) alle leer sein.
Ein Mausklick auf einen bereits hinzugefügten Wegpunkt hat nun zur Folge, dass Linien zu den direkten Nachbarn
gezeichnet werden.
Grüne Linien bedeuten, dass diese Nachbarn nicht mit Objekten kollidieren.
Rote Linien verlaufen durch Objekte oder Wände.
Unter den Kästchen befinden sich 2 Knöpfe, die zum einem den Roboter mit einem lila Kreis vom Start- bis
zum Endpunkt fahren lässt, falls ein Pfad existiert (\emph{Animate}) und eine Statistik ausgibt, worin
die Programmlaufzeiten für verschiedene Berechnungen hinterlegt sind (\emph{Stats}).

\subsection{Szene}

Die Szene dient zur Veranschaulichung und zum Testen der entwickelten Lösung.
In der Szene wird das Bild des Raumes als Hintergrund (Textur) hinterlegt, worauf dann verschiedene Zeichnungen
durchgeführt werden.
Dargestellt wird immer mindestens ein Start- und Endpunkt.
Zusätzliche Einblendungen können über die Bedienoberfläche \ref{Bedienoberfläche} vorgenommen werden.

\subsection{Wegfindung}

Um diese Polygone zu ermitteln, wird eine \emph{Constrained Delaunay Triangulation} (kurz: \emph{CDT})
verwendet.
Bei einer Constrained Delaunay Triangulation 

\section{Handbuch zur Graphischen Bedienoberfläche}

\section{Probleme und Ausblick}

\section{Persönliches Fazit}

\section{Quellen}

\end{document}
